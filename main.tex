\documentclass[a4paper]{report}

%====================== PACKAGES ======================

\usepackage[french]{babel}
\usepackage[utf8x]{inputenc}
%pour gérer les positionnement d'images
\usepackage{float}
\usepackage{amsmath}
\usepackage{graphicx}
\usepackage[colorinlistoftodos]{todonotes}
\usepackage{url}
%pour les informations sur un document compilé en PDF et les liens externes / internes
\usepackage{hyperref}
%pour la mise en page des tableaux
\usepackage{array}
\usepackage{tabularx}
%pour utiliser \floatbarrier
%\usepackage{placeins}
%\usepackage{floatrow}
%espacement entre les lignes
\usepackage{setspace}
%modifier la mise en page de l'abstract
\usepackage{abstract}
%police et mise en page (marges) du document
\usepackage[T1]{fontenc}
\usepackage[top=2cm, bottom=2cm, left=2cm, right=2cm]{geometry}
%Pour les galerie d'images
\usepackage{subfig}
\usepackage{enumitem}
\usepackage[acronym]{glossaries}
\usepackage[toc,page,header]{appendix}
\usepackage{minitoc}

%====================== INFORMATION ET REGLES ======================

%rajouter les numérotation pour les \paragraphe et \subparagraphe
\setcounter{secnumdepth}{4}
\setcounter{tocdepth}{4}

\hypersetup{							% Information sur le document
pdfauthor = {Premier Auteur,
			Deuxième Auteur,
			Troisième Auteur,
    		Quatrième Auteur},			% Auteurs
pdftitle = {Nom du Projet -
			Sujet du Projet},			% Titre du document
pdfsubject = {Mémoire de Projet},		% Sujet
pdfkeywords = {Tag1, Tag2, Tag3, ...},	% Mots-clefs
pdfstartview={FitH}}					% ajuste la page à la largueur de l'écran
%pdfcreator = {MikTeX},% Logiciel qui a crée le document
%pdfproducer = {}} % Société avec produit le logiciel

%============================ VARIABLES ============================
\newcommand{\titre}{Titre}
\newcommand{\anneeAcad}{2021-2022}
\newcommand{\prenom}{Prénom}
\newcommand{\nom}{NOM}
\newcommand{\orientationEtude}{Informatique}
\newcommand{\motsCles}{mots clés}
%======================== DEBUT DU DOCUMENT ========================

%Pour comprendre comment marche gloassaries --> https://fr.overleaf.com/learn/latex/Glossaries

\makeglossaries

\newglossaryentry{mot}
{
        name=mot,
        description={définition du mot en question}
}

\newglossaryentry{2eme mot}
{
        name=2eme mot,
        description={explication du 2eme mot}
}



\newacronym{ecam}{ECAM}{Ecole Centrale des Arts et Métiers}

\newacronym{it}{IT}{Intelligence Technology - Techologie de l'Information}
\begin{document}

%régler l'espacement entre les lignes
\newcommand{\HRule}{\rule{\linewidth}{0.5mm}}

%page de garde
\begin{titlepage}
\begin{center}



% Upper part of the page. The '~' is needed because only works if a paragraph has started.
\Large \textbf{Haute Ecole ICHEC - ECAM - ISFSC}~\\[1.0cm]
\includegraphics[width=0.45\textwidth]{images/logo.jpg}\\[1cm]



\textsc{\Large }\\[2.9cm]

% Title

\fbox{
\begin{minipage}{\textwidth}
\begin{center}
    {\huge \bfseries \titre }
\end{center}
\end{minipage}}

\textsc{\Large }\\[3.5cm]

\begin{flushright}
    Travail de fin d'études présenté par\\[0.5cm]
    \prenom \: \textsc{\nom}\\[0.5cm]
    En vue de l'obtention du diplôme de\\[0.5cm]
    Master en Sciences de l'Ingénieur Industriel orientation \orientationEtude
    
    
\end{flushright}
% Author and supervisor
% \begin{minipage}{0.4\textwidth}
% \begin{flushleft} \large
% \emph{Auteur:}\\
% Premier \textsc{Auteur}\\
% Deuxième \textsc{Auteur}\\
% Troisième \textsc{Auteur}\\
% Quatrième \textsc{Auteur}
% \end{flushleft}
% \end{minipage}
% \begin{minipage}{0.4\textwidth}
% \begin{flushright} \large
% \emph{Client:} \\
% Prénom \textsc{Nom}\\
% \emph{Référent:} \\
% Prénom \textsc{Nom}
% \end{flushright}
% \end{minipage}
\vfill

% Bottom of the page
{\large Année académique \anneeAcad}

\end{center}
\end{titlepage}
\pagenumbering{Roman}
%page blanche
\newpage
~
%ne pas numéroter cette page
\thispagestyle{empty}
\newpage
\input{./remerciements.tex}
\renewcommand{\abstractnamefont}{\normalfont\Large\bfseries}
\renewcommand{\abstractname}{Résumé}
%\renewcommand{\abstracttextfont}{\normalfont\Huge}


\begin{abstract}
\hskip7mm

\begin{spacing}{1.3}

Pour montrer les acronymes voici quelques exemples:
\acrshort{ecam}\\
\acrlong{ecam}\\
\acrfull{ecam}\\
\acrfull{it}\\
Pour montrer quelques exemples sur les définitions(glossay): \Gls{mot}\\
\gls{mot}\\
\Glspl{mot}\\
\glspl{2eme mot}\\
Lorem ipsum dolor sit amet, consectetur adipiscing elit. Sed non risus. Suspendisse lectus tortor, dignissim sit amet, adipiscing nec, ultricies sed, dolor. Cras elementum ultrices diam. Maecenas ligula massa, varius a, semper congue, euismod non, mi. Proin porttitor, orci nec nonummy molestie, enim est eleifend mi, non fermentum diam nisl sit amet erat. Duis semper. Duis arcu massa, scelerisque vitae, consequat in, pretium a, enim. Pellentesque congue. Ut in risus volutpat libero pharetra tempor. Cras vestibulum bibendum augue. Praesent egestas leo in pede. Praesent blandit odio eu enim. Pellentesque sed dui ut augue blandit sodales. Vestibulum ante ipsum primis in faucibus orci luctus et ultrices posuere cubilia Curae; Aliquam nibh. Mauris ac mauris sed pede pellentesque fermentum. Maecenas adipiscing ante non diam sodales hendrerit. 

\end{spacing}
\end{abstract}

\addcontentsline{toc}{chapter}{Cahier des charges}

\begin{center}
\fbox{\begin{minipage}{0.7\textwidth}
\begin{center}
\begin{minipage}{0.9\textwidth}
\begin{spacing}{1.5}
    \vspace{0.3cm}
    {\Large \bfseries CAHIER DES CHARGES RELATIF au TRAVAIL DE FIN D’ETUDES de }
  
    {\; \large \nom \, \prenom \, inscrit en 5MIN} 
    \vspace{-0.3cm}
\end{spacing}
\end{minipage}
\end{center}
\end{minipage}}
\end{center}
~\\[0.5cm]
\begin{spacing}{1.5}
\noindent
%Année académique
\underline{\textbf{Année académique:}} \anneeAcad \\[0.5cm]
%Titre provisoire
\underline{\textbf{Titre provisoire:}} \titre \\[0.1cm]
%Mots-clés
\underline{\textbf{Mots-clés:}} \motsCles \\[0.7cm]
%Objectifs à atteindre
\underline{\textbf{Objectifs à atteindre:}}\vspace{0.3cm}
\begin{itemize}
    \item Objectif 1
    \item Objectif 2
    .\\.\\.\\
    \item objectif..
\end{itemize}
%Principales étapes
\underline{\textbf{Principales étapes:}}\vspace{0.3cm}
\begin{enumerate}[label=\arabic*)]
    \item Etape 1 
    \item Etape 2\\
    .\\
    .\\
    .\\
    \item Etape...
\end{enumerate}
\end{spacing}
\newpage





\addcontentsline{toc}{chapter}{Table des matières}
\tableofcontents
\addcontentsline{toc}{chapter}{Liste des figures}
\listoffigures
\newpage
\addcontentsline{toc}{chapter}{Glossaires et acronyme}
\printglossary[type=\acronymtype]
\printglossary

% \thispagestyle{empty}
\pagenumbering{arabic}
\setcounter{page}{0}
%ne pas numéroter le sommaire

\newpage

%espacement entre les lignes d'un tableau
\renewcommand{\arraystretch}{1.5}

%====================== INCLUSION DES PARTIES ======================

~
\thispagestyle{empty}
%recommencer la numérotation des pages à "1"
\setcounter{page}{0}
\newpage

\addcontentsline{toc}{chapter}{Introduction}
\chapter*{Introduction}

\section*{Contexte}
On explique le contexte général
\subsection*{Contexte particulier}
On précise le contexte particulier
\subsection*{Restriction sanitaire (COVID 19)}
Facultatif pour année COVID

\section*{Problématique}
Qu'est-ce qu'on va rencontrer comme problématique et ce qu'on va faire pour résoudre celle-ci
\subsection*{Objectifs}
Objectifs fixé (du cahier des charges) mais en phrases et non en tiret
\section*{Structure du document}

\chapter{Analyse de l'existant}

Intro

\section{Partie 1}

Intro

\subsection{Sous-partie 1}

Bla

\subsection{Sous-partie 2}

Bla\\

Transition

\section{Partie 2}

Bla\\

Transition

\section{Bilan récapitulatif}

Voici un tableau (cf. fig. 2.1) récapitulatif de notre analyse de l'existant...\\

%tableau centré à taille variable qui s'ajuste automatiquement suivant la longueur du contenu
\begin{figure}[!h]
\begin{center}
\begin{tabular}{|l|l|l|l|l|}
  \hline
  Solution & Critère 1 & Critère 2 & Critère 3 & Critère 4\\
  \hline
  Solution 1(cf. ref. \cite{cite0}) & Oui & Oui & Oui & Oui \\
  Solution 2(cf. ref. \cite{cite1}) & Oui & Oui & Oui & Non \\
  Solution 3(cf. ref. \cite{cite2}) & Oui (sauf telle chose) & Non & Non & Oui\\
  Solution 4(cf. ref. \cite{cite3}) & Oui& Non & Oui & Non\\
  Solution 5(cf. ref. \cite{cite4}) & Oui (uniquement ceux-ci) & Non & Oui & Non\\
  \hline
\end{tabular}
\end{center}
\caption{Tableau récapitulatif des solutions}
\end{figure}

 
\chapter{Analyse des besoins}

Intro

\section{Besoins fonctionnels}

Après une analyse des besoins fonctionnels du projet, nous avons défini deux sous catégories. D'un côté, les besoins [...], de l'autre, les besoins [...].

\subsection{Sous-partie 1}

Bla

\subsection{Sous-partie 2}

Bla

\newpage

\section{Besoins non-fonctionnels}

Comme précédemment, nous avons choisi de distinguer deux catégories pour les besoins non-fonctionnels. D'une part, nous avons les besoins non-fonctionnels pour les [...], et d'autre part ceux pour [...]. Nous avons aussi pris en compte les contraintes de développement, que nous détaillerons à la fin de cette partie.

\subsection{Sous-partie 1}

Bla\\

Aperçu du rendu souhaité :

% \begin{figure}[!h]
% \begin{center}
% \includegraphics[height=10cm]{besoins/rendu}
% \end{center}
% \caption{Rendu attendu}
% \end{figure}

\subsection{Sous-partie 2}

Bla

\newpage

\section{Développement}

Intro

\subsection{Tâches}

Bla\\


%tableau à taille fixée sur certaines colonnes (param sur la ligne \begin{tabularx}, voir wiki pour plus d'info sur la syntaxe
\begin{figure}[!h]
\begin{center}
\begin{tabularx}{17cm}{|c|p{6cm}|X|}
  \hline
  Priorité & Nom & Raison\\
  \hline
  1 & Tache 1 & Doit être vérifié en premier car sinon [...] \tabularnewline
  2 & Tache 2 & On doit pouvoir [...] \tabularnewline
  3 & Tache 3 & Comme les principales fonctionnalités permettant de tester sont opérationnelles, nous pouvons passer à cette tâche. \tabularnewline
  4 & Tache 4 & Parce que [...] \tabularnewline
  5 & Tache 5 & La tache 5 fait partie des principales [...]. \tabularnewline
  6 & Tache 6 & Dernière fonctionnalité essentielle à mettre en place. \tabularnewline
  7 & Tache 7 & Non-essentiel, mais apporterait un plus au projet. \tabularnewline
  8 & Tache 8 & Non-essentiel, mais apporterait un plus au projet. \tabularnewline
  \hline
\end{tabularx}
\end{center}
\caption{Tableau récapitulatif des tâches}
\end{figure}

\subsection{Tests}

Bla\\

\begin{figure}[!h]
\begin{center}
\begin{tabularx}{17cm}{|p{6cm}|X|}
  \hline
  Fonctionnalité & Test\\
  \hline
  Fonction 1 & Quand [...], vérifier [...]. \tabularnewline
  & Et quand [...], vérifier [...]. \tabularnewline
  Fonction 2 & Vérifier [...]. \tabularnewline
  Fonction 3 & Vérifier [...]. \tabularnewline
  Fonction 4 & Avoir [...]. \tabularnewline
  Fonction 5 & Accéder à [...]. \tabularnewline
   & Vérifier que [...]. \tabularnewline
  Fonction 6 & Accéder à [...]. \tabularnewline
   & Et vérifier [...]. \tabularnewline
  Fonction 7 & Installer [...]. \tabularnewline
   & Vérifier [...]. \tabularnewline
  Fonction 8 & Compter [...]. \tabularnewline
  \hline
\end{tabularx}
\end{center}
\caption{Tableau récapitulatif des tests}
\end{figure}

\chapter{Développements}

Dans cette partie nous cherchons à décrire le développements de tout le processus.

\section{Partie 1}

Très modulable comme partie

\section{Matériel, méthode et outil utilise}

\chapter{Résultats}

\section{Partie 1}

Intro

\subsection{Sous-partie 1}

\paragraph*{Paragraphe 1 (n'apparaitra pas dans l'index)} Bla

\paragraph*{Paragraphe 2} Bla

\paragraph*{Paragraphe 3} Bla

\subsection{Sous-partie 2}

Bla

\subsection{Sous-partie 3}

Bla

\section{Partie 2}

Intro

\subsection*{Sous-partie 1 ('apparaitra pas dans l'index)} Bla

\paragraph*{Paragraphe 1 ('apparaitra pas dans l'index)} Bla

\paragraph*{Paragraphe 2} Bla

\paragraph*{Paragraphe 3} Bla

\newpage

\subsection*{Sous-partie 2}

Bla

%galerie d'image
% \begin{figure}[htp]
%   \centering
%   \subfloat[Première image]{\label{fig:première}\includegraphics[scale=0.8]{resultats/gallerie}}
%   ~ %espace entre deux images sur une même ligne
%   \subfloat[Deuxième image]{\label{fig:deuxième}\includegraphics[scale=0.8]{resultats/gallerie}}
%   ~
%   \subfloat[Troisième image]{\label{fig:troisième}\includegraphics[scale=0.8]{resultats/gallerie}}
%   ~\\ %saute une ligne dans la galerie d'image
%   \subfloat[Quatrième image]{\label{fig:quatrième}\includegraphics[scale=0.8]{resultats/gallerie}}
%   ~
%   \subfloat[Cinquième image]{\label{fig:cinquième}\includegraphics[scale=0.8]{resultats/gallerie}}
%   \caption{Différents screenshots quelque chose, en gallerie}
%   \label{fig:gallerie1}
% \end{figure}

\chapter{Conclusion}

%Rappel du context
Intro / Rappel Contexte

Nous avons donc pu en tirer la problématique suivante :

\begin{center}
\hskip7mm
Problématique du sujet
\end{center}

Bla

Bla\\

Bla\\

%Rappel des résultats
Bla

Bla\\

Bla

Bla

\newpage

%Conclusion/Perspectives
Bla

Bla\\

Bla

%récupérer les citation avec "/footnotemark"
\nocite{*}

%choix du style de la biblio
\bibliographystyle{plain}
%inclusion de la biblio
\bibliography{bibliographie.bib}

%Ne pas numéroter cette partie
\part*{Annexes}
\addcontentsline{toc}{part}{Annexes}

%Rajouter la ligne "Annexes" dans le sommaire

\chapter*{Annexe 1}
\addcontentsline{toc}{chapter}{Annexe 1}

%changer le format des sections, subsections pour apparaittre sans le num de chapitre
\makeatletter
\renewcommand{\thesection}{\@arabic\c@section}
\makeatother

%recommencer la numérotation des section à "1"
\setcounter{section}{0}

Intro

\section{Partie 1}

Bla

\subsection{Sous-partie 1}

Bla

\subsection{Sous-partie 2}

Bla

\subsection{Sous-partie 3}

Bla

\section{Partie 2}

Bla

\subsection{Sous-partie 1}

Bla

\subsection{Sous-partie 2}

Bla

\subsection{Sous-partie 3}

Bla

\pagenumbering{Alph}
\chapter*{Annexe 2}
\addcontentsline{toc}{chapter}{Annexe 2}

%recommencer la numérotation des section à "1"
\setcounter{section}{0}

Intro

\section*{Prérequis}
\addcontentsline{toc}{section}{Prérequis}

Bla

\begin{itemize}
\item item1;
\item item2;
\item item3;
\item item4.
\end{itemize}

Bla

\section{Partie 1}

Bla

\subsection{Sous-parie 1}

Bla

\subsection{Sous-parie 2}

Bla

\section{Partie 2}

\begin{center}
\textsc{Attention !}

\textit{Texte d'avertissement}
\end{center}

Bla

\newpage

\section{Partie 3}

Bla

% \begin{figure}[!ht]
% \begin{center}
% % \includegraphics[height=8cm]{presentation/schema}
% \end{center}
% \caption[schema]{Presentation schema}
% \end{figure}

\paragraph*{Paragraphe 1}
~\\
\hskip7mm

Bla

\paragraph*{Paragraphe 2}
~\\
\hskip7mm

Bla

\paragraph*{Paragraphe 3}
~\\
\hskip7mm

Bla


%voir wiki pour plus d'information sur la syntaxe des entrées d'une bibliographie

\end{document}
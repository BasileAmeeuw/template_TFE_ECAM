\chapter{Analyse des besoins}

Intro

\section{Besoins fonctionnels}

Après une analyse des besoins fonctionnels du projet, nous avons défini deux sous catégories. D'un côté, les besoins [...], de l'autre, les besoins [...].

\subsection{Sous-partie 1}

Bla

\subsection{Sous-partie 2}

Bla

\newpage

\section{Besoins non-fonctionnels}

Comme précédemment, nous avons choisi de distinguer deux catégories pour les besoins non-fonctionnels. D'une part, nous avons les besoins non-fonctionnels pour les [...], et d'autre part ceux pour [...]. Nous avons aussi pris en compte les contraintes de développement, que nous détaillerons à la fin de cette partie.

\subsection{Sous-partie 1}

Bla\\

Aperçu du rendu souhaité :

% \begin{figure}[!h]
% \begin{center}
% \includegraphics[height=10cm]{besoins/rendu}
% \end{center}
% \caption{Rendu attendu}
% \end{figure}

\subsection{Sous-partie 2}

Bla

\newpage

\section{Développement}

Intro

\subsection{Tâches}

Bla\\


%tableau à taille fixée sur certaines colonnes (param sur la ligne \begin{tabularx}, voir wiki pour plus d'info sur la syntaxe
\begin{figure}[!h]
\begin{center}
\begin{tabularx}{17cm}{|c|p{6cm}|X|}
  \hline
  Priorité & Nom & Raison\\
  \hline
  1 & Tache 1 & Doit être vérifié en premier car sinon [...] \tabularnewline
  2 & Tache 2 & On doit pouvoir [...] \tabularnewline
  3 & Tache 3 & Comme les principales fonctionnalités permettant de tester sont opérationnelles, nous pouvons passer à cette tâche. \tabularnewline
  4 & Tache 4 & Parce que [...] \tabularnewline
  5 & Tache 5 & La tache 5 fait partie des principales [...]. \tabularnewline
  6 & Tache 6 & Dernière fonctionnalité essentielle à mettre en place. \tabularnewline
  7 & Tache 7 & Non-essentiel, mais apporterait un plus au projet. \tabularnewline
  8 & Tache 8 & Non-essentiel, mais apporterait un plus au projet. \tabularnewline
  \hline
\end{tabularx}
\end{center}
\caption{Tableau récapitulatif des tâches}
\end{figure}

\subsection{Tests}

Bla\\

\begin{figure}[!h]
\begin{center}
\begin{tabularx}{17cm}{|p{6cm}|X|}
  \hline
  Fonctionnalité & Test\\
  \hline
  Fonction 1 & Quand [...], vérifier [...]. \tabularnewline
  & Et quand [...], vérifier [...]. \tabularnewline
  Fonction 2 & Vérifier [...]. \tabularnewline
  Fonction 3 & Vérifier [...]. \tabularnewline
  Fonction 4 & Avoir [...]. \tabularnewline
  Fonction 5 & Accéder à [...]. \tabularnewline
   & Vérifier que [...]. \tabularnewline
  Fonction 6 & Accéder à [...]. \tabularnewline
   & Et vérifier [...]. \tabularnewline
  Fonction 7 & Installer [...]. \tabularnewline
   & Vérifier [...]. \tabularnewline
  Fonction 8 & Compter [...]. \tabularnewline
  \hline
\end{tabularx}
\end{center}
\caption{Tableau récapitulatif des tests}
\end{figure}